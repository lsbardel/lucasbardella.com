%
%
%%%%%%%%%%%%%%%%%%%%%%%%%%%%%%%%%%%%%%%%%%%%%%%%%%%%%%%%%%%%%%%%%%5
%
%    TIME INTEGRATION
%
%%%%%%%%%%%%%%%%%%%%%%%%%%%%%%%%%%%%%%%%%%%%%%%%%%%%%%%%%%%%%%%%%%5
%
%
\section{Time Integration}
\label{time_integration_nonlinear.section}
\headb{Nonlinear Navier-Stokes solver}{Time integration}
%
 After discretising the governing equations in space, the semi-discrete system of
 coupled ordinary differential equations (ODE) in (\ref{semi_discrete_nl.eq}) 
 is obtained. Such system of coupled ODEs can be expressed in a compact
 form as:

%
\beq
  \frac{d \left(  {\cal V}\sm{I}  {\bf U}\sm{I} \right) }{dt}  
     = {\bf R}\sm{I}\left({\bf U}\right)
\label{semi_discrete_nl_2.eq}
\eeq
%
 where ${\bf R}\sm{I}$ represents the discretised form of the inviscid fluxes,
 viscous fluxes and source terms at node $I$.

 A preconditioned multigrid algorithm has been used as an iterative
 method for calculating both steady-state solutions and
 unsteady time-marching solutions via a fully implicit scheme.
%
%
\subsection{Steady state algorithm}
%
 If a steady state solution is sought, time accuracy is not an issue and
 the time integration can be seen as relaxation method towards steady-state.
 Equation (\ref{semi_discrete_nl_2.eq}) is then written in the following
 form:

%
\beq
  \left[{\bf P}\right]\se{-1}L\sm{\tau} \delta {\bf U}\sm{I} =
  {\bf R}\sm{I}\left({\bf U}\right)
  \label{time_steadystate.eq}
\eeq
%
 As discussed in Appendix \ref{multigrid.chap}, $L\sm{\tau}$ represents
 a multistage Runge-Kutta operator while $\left[{\bf P}\right]\se{-1}$
 represents a preconditioner developed for accelerating steady-state
 calculations. $\delta$ indicate the change over a pseudo time step.
 Such a preconditioned Runge-Kutta relaxation algorithm is used as
 a smoother in an agglomeration multigrid algorithm.
 The principle behind this algorithm is that errors associated with high
 frequencies are damped by the smoother while the errors associated
 with the low frequencies are damped on the coarser grids where these frequencies manifest
 themselves as high frequencies.

 A complete description of the algorithm is given in Appendix \ref{multigrid.chap}.
%
%
%
\subsection{Unsteady time-marching algorithm}
%
 A numerical scheme to solve time-marching unsteady aerodynamics is described.
 The scheme is fully implicit and uses the same preconditioned multigrid algorithm
 developed for steady-state predictions, in order to iteratively invert
 the equations at each physical time step.
 The design objective of the method is unconditional stability. This means that the
 choice of the time-step is based on the physical to be resolved, rather then
 limited by numerical stability, which is particularly important for meshes
 with large variations in size.

 It has been demonstrated that A-stable schemes\footnote{An A-stable scheme is
 stable for all values of the time step of $\Delta t$}
 cannot have an order of accuracy higher then two (Jameson \citeyearNP{Jame:6}).
 The trapezoidal scheme has the smallest truncation error of all second order
 A-stable methods but it becomes undamped for large
 time steps\footnote{This can be demonstrated by performing a stability analysis
 of the 1D convection equation.}.
 Consequently a second order implicit backward difference scheme, which is A-stable
 and damped for $\Delta t \rightarrow \infty$, has been preferred for this work.
 
 A second order implicit backward time integration of (\ref{semi_discrete_nl_2.eq})
 can be expressed as:

%
\beq
  \frac{3\left({\cal V}\sm{I}{\bf U}\sm{I}\right)\se{n+1} -
        4\left({\cal V}\sm{I}{\bf U}\sm{I}\right)\se{n} +
         \left({\cal V}\sm{I}{\bf U}\sm{I}\right)\se{n-1}}
       {2\Delta t}
  = {\bf R}\sm{I}\left({\bf U}\se{n+1}\right)
 \label{Dual_time_stepping_1.eq}
\eeq
%
 where $n$ denotes the physical time level.
 The implicit non-linear system of equations given by
 (\ref{Dual_time_stepping_1.eq}) needs to be solved
 every time-step. 
 Indicating with ${\bf U}\se{l}$ the $l^{th}$ approximation to ${\bf U}\se{n+1}$ and
 with $\delta {\bf U}\sm{I} = {\bf U}\se{l+1} - {\bf U}\se{l}$,
 an iterative equation is constructed, from (\ref{Dual_time_stepping_1.eq}),
 by simply adding a pseudo-time derivative term
 $\left[{\bf P}\right]\se{-1} L\sm{\tau} \delta {\bf U}\sm{I}$
 to the left-hand side. 
 
%
\beq
 \left[{\bf P}\right]\se{-1} L\sm{\tau}\delta{\bf U}\sm{I} +
  \frac{3 {\cal V}\sm{I} \delta {\bf U}\sm{I} +
        3 \left(\cal V {\bf U}\right)\sm{I}\se{l} -
        4 \left(\cal V {\bf U}\right)\sm{I}\se{n}  +
          \left(\cal V {\bf U}\right)\sm{I}\se{n-1}}
     {2\Delta t} =  {\bf R}\sm{I}\left({\bf U}\se{l}\right)
\label{Dual_time_stepping_2.eq}
\eeq
%
 This expression can be written in a compact form,
 similar to (\ref{time_steadystate.eq}), as:

%
\beq
 \left[{\bf P}\right]\se{-1}\sm{imp}L\sm{\tau} \delta{\bf U}\sm{I} =
 {\bf R}\sm{I_{imp}}\left({\bf U}\se{l}\right)
\label{Dual_time_stepping_3.eq}
\eeq
%
 where $\left[{\bf P}\right]\se{-1}\sm{imp}$ and ${\bf R}\sm{I_{imp}}$ are given by

%
\beq
 \left[{\bf P}\right]\se{-1}\sm{imp} &=& \left[{\bf P}\right]\se{-1} +
                                         3\frac{{\cal V}\sm{I}}{2\Delta t}
                                          \left[{\bf I}\right]
 \label{implicit_preconditioner.eq}\\
 {\bf R}\sm{I_{imp}}\left({\bf U}\se{l}\right) &=& {\bf R}\sm{I}\left({\bf U}\se{l}\right) -
 3\frac{{\cal V}\sm{I}}{2 \Delta t} {\bf U}\sm{I}\se{l}
 + {\bf E}\sm{I}\se{n}
 \label{implicit_righthandside.eq}
\eeq
%
 ${\bf E}\sm{I}\se{n}$ involves the portion of the physical time derivative at
 previous time steps and is invariant during the iteration process.

%
\begin{equation}
  {\bf E}\sm{I}\se{n} =
  \frac{4\left({\cal V} {\bf U}\right)\sm{I}\se{n} -
         \left({\cal V} {\bf U}\right)\sm{I}\se{n-1}}{2 \Delta t}
\label{time_invariant.eq}
\end{equation}
%
 The preconditioner $\left[{\bf P}\right]\se{-1}\sm{imp}$ on the left-hand side of
 (\ref{Dual_time_stepping_3.eq}) contains a portion
 of the physical-time derivative as suggested by Melson at al.
 \citeyear{Melson:1} in order to produce a stable numerical scheme
 for any choice of the physical time step $\Delta t$.

 In order to clarify this point, let us consider, for a moment,
 a scalar preconditioner, also called local time step:

%
\beq
  \left[{\bf P}\right]\se{-1} = \frac{{\cal V}\sm{I}}{\sigma \Delta \tau}
                                \left[{\bf I}\right]
  \label{local_time_step.eq}
\eeq
%
 where $\Delta \tau$ is the local time step at node $I$,
 and $\sigma$ represents the CFL number.
 Substituting (\ref{local_time_step.eq}) into (\ref{implicit_preconditioner.eq}),
 one can obtain

%
\beq
  \left[{\bf P}\right]\se{-1}\sm{imp} &=&
  \frac{{\cal V}\sm{I}}{\sigma \Delta \tau_{imp}}\left[{\bf I}\right]\\
  \Delta \tau_{imp} &=& \frac{\Delta \tau}{1+\frac{3}{2}\frac{\Delta \tau}{\Delta t}\sigma}
  \label{local_time_step_2.eq}
\eeq
%
 From (\ref{local_time_step_2.eq}), it is apparent that the effect of the
 implicit treatment is to reduce the time step in regions
 of the flow where the ratio of pseudo/physical time steps,
 $\frac{\Delta \tau}{\Delta t}$ becomes large.
 For low-frequency unsteady problems, accuracy constraints permit a
 large physical time step $\Delta t$, so that the time step ratio
 $\frac{\Delta \tau}{\Delta t}$, is generally very small throughout most of
 the computational domain, and the implicit treatment plays a minimal role.
 However, near the far-field, large mesh cell will produce
 correspondingly large pseudo-time stability limits, $\Delta \tau$, and the
 implicit treatment becomes a useful mechanism to ensure stability.
