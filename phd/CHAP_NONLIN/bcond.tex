%
%
%
 For convection dominated phenomena, such as those described by the 
 compressible Navier-Stokes equations, the formulation of correct boundary
 conditions is extremely important and its impact on the numerical scheme
 is often dominating.
 The reason for this strong influence can be traced back to the physical nature of
 the convection propagation phenomena (Hirsh \citeyearNP{Hirsch:1}).
 If all the variables were known at a boundary from the knowledge of the
 physical input, there would be no difficulty; however this is generally not the
 case with hyperbolic systems of partial differential equations.
 The number of physical
 variables that can be freely imposed at a boundary is dependent on the propagation
 properties of the system and, in particular, on the information propagated from
 the boundary towards the inside of the flow region. These are known as
 {\em physical boundary conditions}. The remaing variables will depend on the
 details of the flow and are, therefore, part of the solution. However
 from a numerical point of view, information about all variables
 is required  at the boundary. This additional information gives rise to
 {\em numerical boundary conditions} (Hirsch \citeyearNP{Hirsch:1}).
 
 The presence of viscosity and heat conduction
 transforms the conservation laws of momentum and energy into second-order
 partial differential equations.
 The system of Navier Stokes equations is therefore an hybrid system consisting of
 parabolic momentum and energy equations and of a hyperbolic continuity equation.
 A direct consequence is the need for a greater number
 of physical boundary conditions when dealing with viscous flows instead of
 inviscid flows.
%
%
\paragraph{Flow-tangency.}
%
 Also known as the inviscid wall condition, this boundary condition at the solid walls
 is expressed by the requirement that there is no flow through the surface
 of the moving wall. Only one physical boundary condition can be imposed
 at this boundary because only the forward travelling acoustic wave is entering
 the domain. Mathematically, this condition is expressed by the vanishing of the
 velocity normal to the wall.

%
\beq
  \vec{u}\cdot\vec{n} = 0
  \label{flow_tangency1.eq}
\eeq
%
 Expressed in a {\em week sense}, i.e. through the fluxes, (\ref{flow_tangency1.eq})
 becomes:

%
\beq
 \oint\sm{wall}\vec{\bf F}d{\cal S} =
 \oint\sm{wall}\vec{\bf F}{\scriptstyle p}d{\cal S} =
 \left[
 \begin{array}{c}
 0 \\ p\ n\sm{1} d{\cal S}\\
      p\ n\sm{2} d{\cal S}\\
      p\ n\sm{3} d{\cal S}\\
      p\ \left(\vec{\Omega}\times\vec{x}+\frac{d \vec{x}}{d t}\right)\cdot\vec{n}d{\cal S}
 \end{array}
 \right]
  \label{flow_tangency2.eq}
\eeq
%
\paragraph{No-slip condition.}
%
%
 The no-slip condition at the solid walls is expressed by the requirement that the
 fluid velocity relative to the moving wall must be zero.
 Mathematically this can be expressed as:

%
\beq
  \vec{u} = \vec{v} - \vec{\Omega}\times\vec{x} - \frac{d \vec{x}}{d t} = 0
  \label{no_slip1.eq}
\eeq
%
 In addition to (\ref{no_slip1.eq}), an additional information is required
 due to the presence of the second derivative terms in the Navier-Stokes
 equations\footnote{The no-slip condition is valid for viscous flow only.}.
 Here this additional information is for an adiabatic wall

%
\beq
  \vec{n}\cdot\nabl T = 0
  \label{no_slip2.eq}
\eeq
%
%
\paragraph{Inflow and outflow boundaries.} 
%
 The quasi-3D non-reflecting boundary
 conditions developed by Saxer \& Giles \citeyear{Giles:7} are used
 for steady-state computations.
 These boundary conditions are obtained from a Fourier mode analysis
 of the linearised Euler equations at the far-field boundaries.
 Implemented in a turbomachinery environment, the approach assumes
 that the solution at the boundary is circumferentially decomposed into Fourier modes,
 the $0\se{th}$ mode corresponding to the averaged solution.
 The $0\se{th}$ mode is treated according to the standard 
 1D boundary condition (Thompson \citeyearNP{Thompson:1,Thompson:2})
 which allows the user to specify certain physical quantities at the boundaries.
 Such averaged quantities are:
%
\begin{itemize}
 \item
  {\bf Inflow:} stagnation temperature, stagnation pressure and flow angles.
 \item
  {\bf Outflow:} static pressure.
\end{itemize}
%
 The remaining part of the solution, represented by the sum of the harmonics,
 is treated according to the exact\footnote{The term exact refers to the
 solution of the linear problem. Since the linear problem is itself an approximation
 to the nonlinear one, there will be errors which are proportional to the square
 of the amplitude of the non-uniformities at the far-field boundaries.}
 2D theory of Giles \citeyear{Giles:5,Giles:6}.
 Since this method considers radial flow variations in the $0\se{th}$ mode only, it is
 called quasi-3D non-reflecting boundary conditions. In the absence of any radial
 variations, the boundary conditions are exact within
 the 2D linear theory.

 The  standard 1D characteristic boundary conditions of
 Thompson \citeyear{Thompson:1,Thompson:2} are used for 
 unsteady time-marching aerodynamics.
%
%
% 
\paragraph{Periodic boundaries.}
%
 Taking advantage of the edge-based data structure, the periodicity is handled in 
 a straightforward way as long as the points in the two periodic
 boundaries are located at same axial and radial coordinates.
%
\beq
  {\bf U}_{\theta_0 + \Delta \theta} = {\bf U}_{\theta_0}
\eeq
%
